\section{Wiring}

\begin{center}
    \begin{tabular}{ | l | l | }
    \hline
    \textbf{Ultimate GPS Breakout v3} & \textbf{Raspberry Pi 3} \\ \hline
    VIN & Any 3V3 or 5V \\ \hline
    GND & Any ground \\ \hline
    TX & Pin 10 | GPIO 15 | UART0\_RXD \\ \hline
    RX & Pin  8 | GPIO 14 | UART0\_TXD \\ \hline
    PPS & Pin 12 | GPIO 12 | PCM\_CLK \\ \hline
    \end{tabular}
\end{center}

\section{GPS configuration}

\begin{lstlisting}[label=lst:gps, language=bash, caption=GPS configuration]
# install requirements
apt-get install picocom pps-tools
apt-get install gpsd gpsd-clients python-gps

vim /boot/config.txt
### remove lines:
console=serial0,115200
### add lines:
# gps
dtoverlay=pps-gpio,gpiopin=18
enable_uart=1

echo "pps-gpio" >> /etc/modules

# configure pps dev files
echo -n '' > /etc/udev/rules.d/10-pps.rules
echo 'KERNEL=="ttyS0", SYMLINK+="gps0"' >> /etc/udev/rules.d/10-pps.rules
echo 'KERNEL=="pps0", OWNER="root", GROUP="tty", MODE="0660", SYMLINK+="gpspps0"' >> /etc/udev/rules.d/10-pps.rules

# configure gpsd
echo -n '' > /etc/default/gpsd
echo 'START_DAEMON="true"' >> /etc/default/gpsd
echo 'GPSD_OPTIONS="-n"' >> /etc/default/gpsd
echo 'DEVICES="/dev/gps0"' >> /etc/default/gpsd
echo 'USBAUTO="false"' >> /etc/default/gpsd

# compile ntp with pps and gps support
cd /usr/src
wget http://archive.ntp.org/ntp4/ntp-4.2/ntp-4.2.8p8.tar.gz
tar -zxvf ntp-4.2.8p8.tar.gz
cd ntp-4.2.8p8
./configure
make && make install
\end{lstlisting}
