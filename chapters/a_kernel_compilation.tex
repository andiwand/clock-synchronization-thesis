\begin{lstlisting}[label=lst:kernel, language=bash, caption=Kernel compilation]
# create kernel directory
mkdir -p tmp/boot
mkdir -p tmp/lib/modules

# get the cross compilation toolchain
git clone https://github.com/raspberrypi/tools.git
export ARCH=arm
export CROSS_COMPILE=$(pwd)/tools/arm-bcm2708/gcc-linaro-arm-linux-gnueabihf-raspbian/bin/arm-linux-gnueabihf-
export INSTALL_MOD_PATH=$(pwd)/tmp/lib/modules

# get the kernel source
git clone https://github.com/raspberrypi/linux.git
cd linux
git checkout rpi-4.6.y

# for PREEMPT_RT continue here
# first have a look at the Makefile. VERSION, PATCHLEVEL, and SUBLEVEL define the kernel version.
wget https://www.kernel.org/pub/linux/kernel/projects/rt/4.6/patch-4.6.7-rt14.patch.gz
zcat patch-4.6.7-rt14.patch.gz | patch -p1
rm patch-4.6.7-rt14.patch.gz
# resolve any upcoming conflicts
# PREEMPT_RT done

# configure the kernel
export KERNEL=kernel7
make bcm2709_defconfig
# enable CONFIG_PREEMPT_RT_FULL and HIGH_RES_TIMERS
make menuconfig

# compile
make zImage
make modules
make dtbs
make modules_install

# pack the kernel
cd ..
linux/scripts/mkknlimg linux/arch/arm/boot/zImage tmp/boot/$KERNEL.img
cp linux/arch/arm/boot/dts/*.dtb tmp/boot/
cp -r linux/arch/arm/boot/dts/overlays tmp/boot
cd tmp
tar -czf ../kernel.tar.gz *
\end{lstlisting}

\section{Kernel module compilation}

\begin{lstlisting}[label=lst:kernel_module, language=bash, caption=Kernel module compilation]
git clone https://github.com/andiwand/gpio-netlink.git
cd gpio-netlink
# set the following variables appropriately
# CCPREFIX is used for cross compilation
# KERNEL_SRC points to the compiled kernel source directory
CCPREFIX=$(pwd)/../tools/arm-bcm2708/gcc-linaro-arm-linux-gnueabihf-raspbian/bin/arm-linux-gnueabihf- KERNEL_SRC=$(pwd)/../linux make
\end{lstlisting}
